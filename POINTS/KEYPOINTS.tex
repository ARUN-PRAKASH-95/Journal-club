\documentclass[12pt]{article}
\title{Keypoints}
\begin{document}
\maketitle
\section{Basics}
\noindent 1.What is a Dislocation avalanche?\\
 \indent   In macroscopic scale the plastic deformation appears as smooth curve, but in microscopic scale the deformation happens intermittently i.e it takes place through bursts. This caused due to dislocation avalanches'
\\  
\\
2. Cause of dislocation avalanches?\\
\indent Crystal deform by dislocation multiplication(due to frank reed source or single arm dislocation source) and avalanches. Avalanche starts with dislocation accumalation and formation of dislocation bands. The dislocation gets piled up before the bands whose give away causes avalanches(like opening a flood gate). In this case instantaneous strain rate exceeds average strain rate by several orders of magnitude.\\
\\
3. Measurement techniques?\\
\indent Dislocation avalanches generate acoustic waves which can be used to describe the properties of the avalanches

\section{1st Paper}
1.\indent Experiments are mostly carried out on ice Ih a hexagonal closed packed crystal with strong anisotropy because the transparency of ice allows to verify that AE is not related to microcracking.
\\
2.\indent  Only uniaxial compression is carried out by changing both  temperature and applied stress \\
3.\indent One to six piezoelectric transducers of frequency bandwidth 200-750 kHz are fixed to the side of the cylindrical sample. The dynamic range of the amplitude threshold was set between 30db and 100dB. For each event above threshold the AE acquisition system will measure different parameters like maximum amplitude,avalanche duration and arrival time(time at which the waves go above range).Duration between end of an event and possibility of another event is given to avoid secondary waves. The wave measured is then compared with average duration of acoustic wave having same A maximum.
\subsection{Effect of temperature}
1.\indent To find the effect of temperature on the behaviour of dislocations uniaxial compression creep tests are carried out at -3,-10 and $-20 ^o C$. The resolved shear stress on the basal plane is set as function of temperature and the average velocity of single dislocation is kept constant. Whatever the method used the avalanche size distribution was found to be independent of the temperature.\\
2. \indent The thermally activated process like slip and climb acts only on individual dislocations but not on their collective dynamics.\\
\subsection{Effect of grain boundaries and strain hardening}
1.\indent In polycrystals GBs will act as barrier to dislocation motion that will generate internal stress. This internal stress will activate the dislocation source in neighbouring grains.\\
2.\indent So the GB barrier will hinder large scale propagation of dislocation avalanches. The distibution of avalnche sizes of polycrystal compared with single crystal differed in two ways-power law exponent is smaller and there is a cut off of power law scaling observed at large amplitudes\\
3.\indent The sudden development of internal stress due to GB interaction is then forced to relax by creating an aftershock into neighbouring grain\\
4.\indent The slowing down of dislocation motion is a result of hardening of the material(Kinematic hardening)
\subsection{Conclusion}
1.\indent Temperature was found to have no effect on the avalanche distribution of single crystals but it affects the duration of avalanches. Higher the temperature shorter the avalanche duration\\
2.\indent In polycrystals GB acts as barrier to propagation of avalanches and also transmit internal stress\\
3.\indent The duration of relaxation was decreased by GBs because of hardening\\
4.\indent The avalanche durations are influenced by both phonon drag and strain hardening
\section{2nd Paper}
1.\indent Dislocation avalanches follow a power law distribution between no of bursts and their magnitude.\\
2.\indent In this paper dislocation motions inside a slowly compressed high entropy nanopillars is follwed using TEM and picoindenter.
\section{self-organised criticality}
\indent The applications of SOC go well beyond granular piles but the basic picture remains the same: many slowly driven nonequilibrium systems organize in a poised state -- the critical state -- where anything can happen within well-defined statistical laws.
\end{document}
